% !TEX TS-program = xelatex
% !TEX encoding = UTF-8 Unicode
% !Mode:: "TeX:UTF-8"

\documentclass{resume}
\usepackage{zh_CN-Adobefonts_external} % Simplified Chinese Support using external fonts (./fonts/zh_CN-Adobe/)
%\usepackage{zh_CN-Adobefonts_internal} % Simplified Chinese Support using system fonts
\usepackage{linespacing_fix} % disable extra space before next section
\usepackage{cite}
\usepackage{graphicx}	
\usepackage{amssymb}
\begin{document}
\pagenumbering{gobble} % suppress displaying page number

\name{申恒恒}

% {E-mail}{mobilephone}{homepage}
% be careful of _ in emaill address
\contactInfo{heng960509@gmail.com}{(+86) 183-1777-4480}{http://rh01.github.io}
% {E-mail}{mobilephone}
% keep the last empty braces!
%\contactInfo{xxx@yuanbin.me}{(+86) 131-221-87xxx}{}
 
\section{\faGraduationCap\  教育背景}
\datedsubsection{\textbf{安阳工学院}, 安阳}{2013 -- 至今}
\textit{在读本科生}\ 网络工程, 预计 2017 年 6 月毕业


\section{\faUsers\ 实习/项目经历}
\datedsubsection{\textbf{成都琛石科技有限公司}  }{2015年3月 -- 2015年5月}
\role{在校实习生}{}
负责校园推广业务和在线课程评测任务

\datedsubsection{\textbf{纸币清分机}}{2016年4月 -- 2016年5月}
\role{Pythonlang, raspberry, Linux}{个人项目,和李中泽,崔明阳等合作开发}
\begin{onehalfspacing}
利用机器学习技术结合Opencv实现实时纸币分类器
\begin{itemize}
  \item 利用PCA和神经网络实现纸币分类器
  \item 使用OpenCV对图像实时处理和分析
  \item 利用分布式架构对数据处理进行分布计算
  \item 利用了物联网设备Raspberry,并在该平台进行编程
  \item 相比市场上现有的模块,成本较低,较为创新
  \item 不足的为识别率约为75\%
\end{itemize}
\end{onehalfspacing}

\datedsubsection{\textbf{RC-Car自动驾驶}}{2016年5月 -- 至今}
\role{单片机, Linux, Raspberry, Neural Network, Python}{个人项目,和李中泽合作开发}
\begin{onehalfspacing}
自动驾驶 Model
\begin{itemize}
  \item 利用神经网络算法训练模型
  \item 结合OpenCV对图像实时分析和处理
  \item 在单片机上对硬件进行编程控制
\end{itemize}
\end{onehalfspacing}

\datedsubsection{\textbf{P2P分布式文件共享系统}}{2016年6月7日 -- 2016年6月17日}
\role{Python, P2P, 分布式}{个人项目}
\begin{onehalfspacing}
P2P分布式文件系统, https://github.com/rh01/p2p
\begin{itemize}
  \item P2P集中式体系架构
  \item 添加日志功能
  \item 使用多线程编程模型
\end{itemize}
\end{onehalfspacing}

% Reference Test
%\datedsubsection{\textbf{Paper Title\cite{zaharia2012resilient}}}{May. 2015}
%An xxx optimized for xxx\cite{verma2015large}
%\begin{itemize}
%  \item main contribution
%\end{itemize}

\section{\faCogs\ IT 技能}
% increase linespacing [parsep=0.5ex]
\begin{itemize}[parsep=0.5ex]
  \item 编程语言: C == Python == Matlab > C++ > R 
  \item 排版系统: \LaTeXe, $\mathbb{C}$\TeX
  \item 平台: Linux, Windows
  \item 大数据平台: Hadoop, Spark
  \item Web框架: Flask, Django
  \item 全栈开发: angularJS,Bootstrap, NodeJS
  \item 开发: FTP服务器与客户端实现(大二项目),嗅探器sniffer(大二项目),目标跟踪系统,书写数字识别系统,人脸识别系统,基于Web的图像搜索引擎(Docker应用),推荐系统(正在开发中),基于Flask的博客系统(已部署到云端)等,部分应用可以在已托管在github上.

\end{itemize}

\section{\faCreditCard\ 资格认证(有关机器学习,计算机视觉方面的)}
\datedsubsection{\textbf{Machine Learning}}{2016年1月}
%\role{Stanford University, Andrew Ng}{}
\begin{itemize}
  \item 认证机构: Coursera Course Certificates
  \item 证书编号: QJDPM9F9XNL4
  \item 认证链接: https://www.coursera.org/account/accomplishments/verify/QJDPM9F9XNL4
\end{itemize}
\datedsubsection{\textbf{Machine Learning: Regression}}{2016年1月}
% \role{Coursera}{}
\begin{itemize}
  \item 认证机构: Coursera Course Certificates
  \item 证书编号: HC4DW3KJTUND
  \item 认证链接: https://www.coursera.org/account/accomplishments/verify/HC4DW3KJTUND
\end{itemize}
\datedsubsection{\textbf{Image and video processing}}{2016年3月}
% \role{Coursera}{}
\begin{itemize}
  \item 认证机构: Coursera Course Certificates
  \item 证书编号: 4FS2GFDC78
  \item 认证链接: https://www.coursera.org/account/accomplishments/verify/4FS2GFDC7
\end{itemize}

\datedsubsection{\textbf{Machine Learning: Classification}}{2016年4月}
% \role{Coursera}{}
\begin{itemize}
  \item 认证机构: Coursera Course Certificates
  \item 证书编号: 43GVTKJ5RKTQ
  \item 认证链接: https://www.coursera.org/account/accomplishments/verify/43GVTKJ5RKTQ
\end{itemize}

\section{\faHeart\ 获奖情况}
\datedline{\textit{省级一等奖}, 全国大学生数学建模大赛}{2015 年9 月}
\datedline{其他奖项}{2014}

\section{\faInfo\ 其他}
% increase linespacing [parsep=0.5ex]
\begin{itemize}[parsep=0.5ex]
  \item 技术博客: http://rh01.github.io
  \item GitHub: https://github.com/rh01
  \item Linkdin: https://www.linkedin.com/in/heng960509
  \item 语言: 英语 - 中等
\end{itemize}

\section{\faSearch 照片}
\begin{figure}[!htbp]
	\begin{flushleft}
	\includegraphics[scale=0.8]{photo}
	\end{flushleft}
\end{figure}

%% Reference
%\newpage
%\bibliographystyle{IEEETran}
%\bibliography{mycite}
\end{document}