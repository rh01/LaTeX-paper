%%%%%%%%%%%%%%%%%%%%%%%%%%%%%%%%%%%%%%%%%
% Weekly Timetable Calendar
% LaTeX Template
% Version 1.1 (4/6/13)
%
% This template has been downloaded from:
% http://www.LaTeXTemplates.com
%
% Original calendar style author:
% Evan Sultanik (http://www.sultanik.com/LaTeX_calendar_style)
%
% License:
% CC BY-NC-SA 3.0 (http://creativecommons.org/licenses/by-nc-sa/3.0/)
%
% Important note:
% This template requires the calendar.sty file to be in the same directory as the
% .tex file. The calendar.sty file provides the necessary structure to create the
% calendar.
%
%%%%%%%%%%%%%%%%%%%%%%%%%%%%%%%%%%%%%%%%%

%----------------------------------------------------------------------------------------
%	PACKAGES AND OTHER DOCUMENT CONFIGURATIONS
%----------------------------------------------------------------------------------------

\documentclass[landscape,a4paper]{article}

\usepackage{calendar} % Use the calendar.sty style
\usepackage{setspace}

\usepackage[landscape,margin=1.0in]{geometry}
\usepackage[encapsulated]{CJK}
\begin{document}

\pagestyle{empty} % Removes the page number from the bottom of the page

\noindent

\StartingDayNumber=1 % Calendar starting day, default of 1 means Sunday, 2 for Monday, etc

%----------------------------------------------------------------------------------------
%	TITLE SECTION
%----------------------------------------------------------------------------------------
\begin{CJK}{UTF8}{gkai}
\begin{center}
\textsc{\LARGE Learning-Assignment\large s}\\ % Title text
\begin{spacing}{1.5}
\end{spacing}
\textsc{\large 第一周任务安排} % Subtitle text
\end{center}
\end{CJK}
%----------------------------------------------------------------------------------------
\begin{CJK}{UTF8}{gbsn}
\begin{calendar}{\hsize}

%----------------------------------------------------------------------------------------
%	FIRST DAY
%----------------------------------------------------------------------------------------

\day{}{\textbf{8am-10am} \daysep CUTU-数理统计学\\[3pt]
\textbf{10am-12pm} \daysep 高等数学-多元函数微分学 \\[3pt]
\textbf{1pm- 2pm} \daysep 考研英语精读(94-04)\\[3pt]
\textbf{2:10pm- 3:10pm\\3:20- 4:20pm}\daysep 多元函数微分学-做题巩固 \\[3pt]
\textbf{4:30pm- 5:30pm} \daysep 多元函数微分学总结 \\[3pt]
\textbf{7pm- 8pm} \daysep 英语单词记忆 \\[3pt]
\textbf{8:20pm- 10:20pm} \daysep EM-algorithm{\&}PCA
} % By default all daily events are centered in the box, in order to bring them up use \vspace{2cm} after the event text; you may need to change the 2cm

%----------------------------------------------------------------------------------------
%	SECOND DAY
%----------------------------------------------------------------------------------------

\day{}{
\textbf{8am- 10am} \daysep 高等数学-二重积分 \\[3pt]
\textbf{10am- 12pm} \daysep CUTU-数理统计学 \\[3pt]
\textbf{1pm- 2pm} \daysep 考研英语精读(94-04) \\[3pt]
\textbf{2:10pm- 3:10pm} \daysep 多元函数微分学-做题巩固 \\[3pt]
%\textbf{1pm-2pm} \daysep No Lecture \\[3pt]
\textbf{3:20pm- 5:20pm} \daysep 一元函数积分学-复习 \\[3pt]
\textbf{7pm- 8pm} \daysep 英语单词记忆 \\[3pt]
\textbf{8pm- 10pm} \daysep Classification Course
} 

%----------------------------------------------------------------------------------------
%	THIRD DAY
%----------------------------------------------------------------------------------------

\day{}{ % Tuesday
\textbf{8am- 10am} \daysep 高等数学-二重积分 \\[3pt]
\textbf{10am- 12pm} \daysep CUTU-数理统计学 \\[3pt]
\textbf{1pm- 2pm} \daysep 考研英语精读(94-04) \\[3pt]
\textbf{2:10pm- 3:10pm} \daysep 多元函数微分学-做题巩固 \\[3pt]
%\textbf{1pm-2pm} \daysep No Lecture \\[3pt]
\textbf{3:20pm- 5:20pm} \daysep 一元函数积分学-复习 \\[3pt]
\textbf{7pm- 8pm} \daysep 英语单词记忆 \\[3pt]
\textbf{8pm- 10pm} \daysep Assignments{\&}Quiz-IIPP
} 

%----------------------------------------------------------------------------------------
%	FOURTH DAY
%----------------------------------------------------------------------------------------

\day{}{ % Wednesday
\textbf{8am- 10am} \daysep 高等数学-微分方程 \\[3pt]
\textbf{10am- 12pm} \daysep CUTU-数理统计学 \\[3pt]
\textbf{1pm- 2pm} \daysep 考研英语精读(94-04) \\[3pt]
\textbf{2:10pm- 3:10pm\\3:20pm- 4:20pm} \daysep 二重积分-做题巩固 \\[3pt]
%\textbf{1pm-2pm} \daysep No Lecture \\[3pt]
%\textbf{3:20pm- 5:20pm} \daysep 一元函数积分学-复习 \\[3pt]
\textbf{7pm- 8pm} \daysep 英语单词记忆 \\[3pt]
\textbf{8pm- 10pm} \daysep Classification-Assignments{\&}Quizs
} 

%----------------------------------------------------------------------------------------
%	FIFTH DAY
%----------------------------------------------------------------------------------------

\day{}{ % Thursday
\textbf{8am- 10am} \daysep 高等数学-微分方程 \\[3pt]
\textbf{10am- 12pm} \daysep CUTU-数理统计学 \\[3pt]
\textbf{1pm- 2pm} \daysep 考研英语精读(94-04) \\[3pt]
\textbf{2:10pm- 3:10pm\\3:20pm- 4:20pm} \daysep 二重积分-做题巩固 \\[3pt]
%\textbf{1pm-2pm} \daysep No Lecture \\[3pt]
\textbf{7pm- 8pm} \daysep 英语单词记忆 \\[3pt]
\textbf{8pm- 10pm} \daysep Images{\&}Videos Segmentation
} 

%----------------------------------------------------------------------------------------
%	SIXTH DAY
%----------------------------------------------------------------------------------------

\day{}{ % Friday
\textbf{8am- 10am} \daysep 二重积分总结 \\[3pt]
\textbf{10am- 12pm} \daysep 单变量微积分复习 \\[3pt]
\textbf{1pm- 2pm} \daysep 考研英语精读(94-04) \\[3pt]
\textbf{2pm- 5pm} \daysep 微分方程解法、题型总结 \\[3pt]
%\textbf{1pm-2pm} \daysep No Lecture \\[3pt]
\textbf{7pm- 8pm} \daysep 英语单词记忆 \\[3pt]
\textbf{8pm- 10pm} \daysep Classification Conclusion By {\LaTeX{}} 
}
%----------------------------------------------------------------------------------------
%	SEVENTH DAY
%----------------------------------------------------------------------------------------

\day{}{
\textbf{8am- 12am} \daysep IIPP(Quizs) for Week5 \\[3pt]
\textbf{1pm- 2pm} \daysep 考研英语精读(94-04) \\[3pt]
%\textbf{1pm-2pm} \daysep No Lecture \\[3pt]
\textbf{2pm- 5pm} \daysep Classfication \\[3pt]
\textbf{7pm- 8pm} \daysep 英语单词记忆 \\[3pt]
\textbf{8pm- 10pm} \daysep Make Plan For Next Week
 % Saturday
}
%----------------------------------------------------------------------------------------
 
\finishCalendar
\end{calendar}
\\

\begin{enumerate}

\item All assginments must be accomplished at last.
\item Seize the day,seize the day,Do it now right away.
\item The time table made by 申恒恒 using \LaTeX{}

\end{enumerate}
\end{CJK}
\end{document}
