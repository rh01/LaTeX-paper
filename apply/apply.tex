\documentclass[UTF8,a4paper]{ctexart}
\usepackage{color}
% \usepackage{listings}
\usepackage{epstopdf}
\usepackage{graphicx}
\usepackage{multirow}
\usepackage{xcolor,graphicx}
\usepackage{tikz}
\usepackage{epigraph}
\usepackage{lipsum}

%\setCJKmainfont[ItalicFont={AR PL UKai CN}]{AR PL UMing CN} %设置中文默认字体
%\setCJKsansfont{WenQuanYi Zen Hei} %设置文泉驿正黑字体作为中文无衬线字体
%\setCJKmainfont{AR PL UMing CN}
% \setCJKmonofont{WenQuanYi Zen Hei Mono} %设置文泉驿等宽正黑字体作为中文打字机字体

\setCJKmainfont{SimSun}
% 设定页边距
\usepackage[top=2.5cm,bottom=2.5cm,left=2.5cm,right=2.5cm]{ geometry}
\usepackage[center]{titlesec}
\usepackage{indentfirst}
\usepackage{graphicx}
%设置首行缩进
\usepackage{indentfirst}
\setlength{\parindent}{2em }
\setlength{\parskip}{0pt }
%段前段后距离设置
\CTEXsetup[beforeskip=0ex]{paragraph}
%页眉和页脚

\usepackage{fancyhdr}
\pagestyle{fancy}
\lhead{Research Motives}
\chead{}
\rhead{}
\lfoot{}
\cfoot{}
\rfoot{}
%\pagenumbering{Roman}


\renewcommand{\headrulewidth}{0pt}
\begin{document}


%设置中文章节的格式的字体
%\CTEXsetup[format={\centering\zihao{3}}]{chapter}
\CTEXsetup[format={\bf\zihao{3}\centering}]{section}
\CTEXsetup[format={\bf\zihao{4}}]{subsection}

\titleformat{\section}{\centering\Large\bfseries}{\S\,\thesection}{1em}{}


%设置中文摘要
%\CTEXoptions[abstractname={\zihao{1}研究动机说明}]
%\chapter{研究动机说明}
\section*{\zihao{3}研究动机说明}
尊敬的常虹老师,您好。我是来自安阳工学院的申恒恒,当看到您在中国科学院计算所的个人主页时,看到您在机器学习、计算机视觉和模式识别领域做出的成果,我很激动,很想成为您的学生,和您在这条机器学习,深度学习的学术路上一直走下去。不瞒您说,我学习机器学习已经持续将近一年半的时间,从当初的懵懂到现在的热爱,一步步使我陷入了机器学习的世界。下面我向您陈述一下我在校的项目与未来的研究方向。\\
\indent 我与机器学习的缘分起源于我在2015年6月参加了中国科学院主办的"2015年第一届人工智能大会",当时很荣幸参加这样的人工智能领域的盛会。当时对于一个大二的我来说,完全是一个陌生的词汇,然而当在持续两天的会议中,我收获颇丰,可以这么说,培养了之后我学习人工智能以及机器学习的信心和做机器学习的动力,当时刚好参加数学建模大赛,在回到学校的期间,就开始着手机器学习系统的学习,当时的我已经是一个MOOC学习者,经常在Coursera,Edx,Udacity,网易公开课甚至YouTube上进行学习,但是当时对机器学习的认识尚浅,对Coursera上的Machine Learning课程不敢去尝试,然而经历了这次机器学习界的大佬门的知识的洗礼后,我开始了机器学习之路,并在该道路上越走越远,越爬信心越大和并且对智慧世界的认识越来越深,值得可说的是,暑假期间,我在校参加全国大学生数学建模比赛的校内集训,当时在Modeling的时候,一边学习数学建模一边学习机器学习,在学习的过程中,将其两者结合起来之后,发现机器学习甚至人工智能算法在数学建模的训练和学习中都能看到影子,神经网络是我当时最感兴趣的一个算法,在决赛时我使用了图像处理的知识-图像畸形矫正和计算机测量技对图像视频的处理,使得在此次比赛中我们小组获得了省级一等奖,相得益彰的是当时数学建模的Matlab编程使得我在机器学习课程如虎添翼,顺利完成了所有的编程任务,但是当结课之后,我发现有很多理论性的知识只是停留在浅层,所以我就开始寻找一个机器学习的一个方向来对机器学习更底层的认识,所以我有大量的充电,买了李航老师的《统计机器学习》,周志华老师的《机器学习》,《神经网络与机器学习ebook》等书来不断的学习,但随着枯燥的理论知识的学习,慢慢的浮躁起来,想加强实践能力的学习来做出智能产品,后来我有在Coursera上注册了台湾大学的林志轩老师的《機器學習基石》,华盛顿大学的机器学习专项课程(目前已完成四门课程)等,其中某些课程里面包含了很多的动手项目,所以在动手的基础上又对机器学习有了数据操作上的认识,而后在今年的全国机械创新大赛和我们学校的机械学院进行了学科之间的交流和合作,在此期间我通过主特征分析加上神经网络算法对纸币数据建立模型,对纸币进行分类,但是由于技术和数据特征上的原因,模型的准确率和召回率并不理想,不够此次比赛我动手参与了机器学习算法在工程上的实践,也使得我清楚地认识了自己在项目中的知识的不牢,尤其在图形图像和模型的优化方面的欠缺,至此开始学习杜克大学的图像处理课程和台湾大学的信号处理等课程,也使得自己在图形图像方面的也不那么迷茫,在铺网式的学习过程中,我对计算机视觉方向和机器学习感兴趣,当然两者的结合更让我看到自己在之后的研究和从事课题。在考研期间学习了CS231课程,同样将计算机视觉和机器学习结合也会成为我大学毕业的毕业项目,计划通过训练机器学习模型将其结合计算机视觉技术完成小车自动驾驶项目。至此我深深的对该领域的发展和技术的不断演进感到极大的欣慰也充满了压力。我认为视觉系统对未来智慧型世界的影响是空前巨厚的,然而实现该手段的主要方法就是对机器学习和模式识别等领域的不断的挖掘和研究,使得计算机在图像处理方面识别率逐步渐进人类的识别能力,因为哪怕1\% 的提升都意味着减少不必要的资源牺牲。学术领域往往是人工智能的先驱者和创造者,我对该领域的研究极为感兴趣,加上自己在本科阶段的学习不足或指导不足,导致学习效率低效,所以我想通过研究生阶段的学习使自己不断地对充实自己,并在机器学习的路上越走也扎实。\\
\indent 我在本科阶段涉猎较广,由于科班出身于网络工程,在系网络实验室进行学习,在此期间学习网络知识,学习linux运维,python开发,网络爬虫,数据挖掘等等,自己制作的项目及参与的项目有基于P2P分布式文件共享平台,基于图片的搜索引擎等等,并且熟练掌握了 \LaTeX 排版等。\\
\indent 下面是我在2017年全国研究生考试的初试成绩单(供您参考):

\begin{table}[hbp]
\centering  % 表居中
\begin{tabular}{lcccc}  % {lcccc} 表示各列元素对齐方式,left-l,right-r,center-c
\hline
思想政治理论 & 英语一& 数学一 & 计算机学科综合(专业)& 总分\\ \hline  % \hline 在此行下面画一横线
55 &56 &121 &106&338 \\ \hline       % \\ 表示重新开始一行

\end{tabular}
\caption{\color{blue}{初试成绩}}\label{初试成绩}
\end{table}

\indent 由于我在本次考试中成绩不好,可能这样的成绩相对牵强,还是挺希望老师能考虑一下,让我尝试一下.也许上面介绍的有些略过简略,有什么问题还希望老师您能提出来.希望能收到你的答复,谢谢.






\end{document}