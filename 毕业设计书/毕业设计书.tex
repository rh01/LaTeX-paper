\documentclass[UTF8,a4paper,cs4size]{ctexart}
\usepackage{color}
\usepackage{listings}
\usepackage{epstopdf}
\usepackage{graphicx}
\usepackage{multirow}
\usepackage{xcolor,graphicx}
\usepackage{tikz}
\usepackage{epigraph}
\usepackage{lipsum}
\usepackage{amssymb}
\setCJKmainfont{SimSun} %设置songti
\setCJKmonofont{SimHei} %设置文泉驿等宽正黑字体作为中文打字机字体

% 设定页边距
\usepackage[top=2.5cm,bottom=2.5cm,left=2.5cm,right=2.5cm]{ geometry}
% 加载 ams 数学公式与数学字体宏包
\usepackage{amsmath, amsfonts}
\usepackage{indentfirst}
\usepackage{graphicx}
\usepackage[colorlinks = true,
            linkcolor = blue,
            urlcolor  = blue,
            citecolor = blue,
            anchorcolor = blue]{hyperref}
\usepackage{svg}
%设置首行缩进
\usepackage{indentfirst}
\setlength{\parindent}{2em }
\setlength{\parskip}{0pt }
%段前段后距离设置
\CTEXsetup[beforeskip=2ex]{paragraph}
%页眉和页脚

\usepackage{fancyhdr}
\pagestyle{fancy}
\lhead{}
\chead{}
\rhead{}
\lfoot{}
\cfoot{\thepage}
\rfoot{}
\pagenumbering{Roman}
%\documentclass[letterpaper]{article}%




\renewcommand{\headrulewidth}{0pt}

\begin{document}
\CTEXsetup[format={\zihao{3}\heiti\centering}]{section}
\CTEXsetup[format={\zihao{4}\heiti}]{subsection}

%设置中文摘要

\section*{Auto-Driven Based on CNN}
\begin{center}
\textit{Shen Hengheng \hspace{10mm} Li Zhongze}
\end{center}
\noindent 产品描述:基于机器学习、计算机视觉、数字图像处理等技术的人工智能产品“自动驾驶小车(改装遥控车)”,该项目由申恒恒(CS)和李中泽(ME)合作开发,指导老师为闫怀平老师。

\noindent 关键技术:Python + OpenCV 神经网络 + Haar 级联分类器
\subsection*{目标}
在车道上自动驾驶,停止(STOP)标志和交通红绿灯的检测与识别,自动检测前方障碍物(比如前方小车),并控制(车速或保持车距)避免与其碰撞或追尾.
\subsection*{系统的设计}
该系统由三个子系统所组成,其中包括输入单元 (相头,超声波传感器)、 处理单元 (计算机) 和 小车的控制单元。
\subsubsection*{输入单元}
树莓派板(型号 B +),附有一个 pi 相机模块,和一个用来收集输入数据的 HC SR04 超声波传感器。树莓派运行两个客户端程序,一个是用来传输视频流,另外一个是通过本地Wi-Fi连接(这里使用树莓派做无线路由)将超声波传感器的数据传输到计算机。为了实现低延迟的视频流。在这里视频缩小为QVGA(320×240)分辨率。
\subsubsection*{处理单元}
处理单元 (计算机) 处理多个任务:接收来自树莓派的数据、训练神经网络和预测(转向的方向),目标检测 (停止标志和交通红绿灯),距离测量 (单目视觉技术),并将指令发送到Arduino,Arduino通过 USB 与计算机连接 。
\subsubsection*{\textit{TCP Server}}
运行在计算机端的多线程TCP服务器,负责接受来自树莓派的视频流图像帧和超声波传感器数据,其中每一图像帧被转化为灰度图像并存储在numpy数组里。
\subsubsection*{\textit{Neural Network}}
使用神经网络的优点是,一旦神经网络训练完毕,它只需要加载之后训练完成后的参数,所以它的预测是非常快的。只有小部分的输入图像被用于训练和预测。在输入层有38,400(320 × 120)个节点,隐含层有 32 个节点。在隐含层节点的数目是选择是随意选择的。在输出层有四个节点,每个节点对应于小车转向的控制指令,左,右,正向和反向分别 (虽然在此项目中,反向不在任何地方使用,但它还包括在输出层)。

下面显示训练数据收集过程。首先每个帧被裁剪并转换为一个 numpy 数组。然后训练图像被标记相应的label (人工标记)。最后,所有标记完成的图像数据和相应标签都保存到npz文件中。在 OpenCV 中使用反向传播算法训练神经网络。训练完成后,生成的神经网络节点的权重被保存为xml文件。若要对新的图像数据生成预测,直接加载xml文件构建与训练数据生成的神经网络相同的网络结构,也可以这么说,训练后的模型被保存在xml文件(级联文件)中,若要预测,直接加载该文件即可。
\subsubsection*{\textit{Object Detection}}
这个项目采用了基于形状检测的方法,并且使用基于特征的 Haar 级联分类器,主要用于目标检测。由于每个目标对象需要其自己的分类器,并在训练过程中和检测过程中是相同,所以这个项目只集中于停止(STOP)标志和交通红绿灯的检测。

OpenCV 提供了训练器以及特征探测器。(包含目标对象) 的正样本图像主要使用了安卓手机拍摄了一些图像,并将图像裁剪,使得正样本中只有目标图像可见。负样本图像 (没有目标对象)随机图像。比如,交通红绿灯正样本包含相同数量的红色交通灯和绿色交通灯。同样的,负样本数据集使用了用于停止的标志和黄色交通灯来进行训练。下面显示了在该项目中使用一些正样本和负样本图像。

为了识别不同的状态的交通红绿灯(红,绿),所以在检测后需要进行必要的图像处理。下面的流程图总结了交通灯识别的过程。

首先,利用训练完成的级联分类器用来检测交通红绿灯的状态。边框被认为是感兴趣区 (ROI)。其次,将高斯模糊(高斯低通滤波器)应用在ROI中,用来减少图像噪音。第三,在ROI在找到最亮的点。最后,通过基于ROI中最亮的点的位置就可以确定交通红绿灯的状态(红,绿)。
\subsubsection*{\textit{Distance Measurement}}
由于树莓派只能支持一个 pi 摄像头模块。另外在计算机视觉领域往往会通过双目视觉技术来测量距离,但是使用两个 USB 网络摄像头会为小车带来额外的重量,所以不符合实际应用。因此,在这里选择树莓派摄像头模块并且使用单目视觉技术。

在本项目中采用了 \href{http://urlc.cn/gWN7Y4}{
Chu Jiangwei,Ji Lisheng,Guo Lie,Libibing,Wang Rongben} 提出的单目视觉几何模型来探测与目标之间的距离。

$P$ 是在目标对象上的一个点;$d$ 为从光学中心到 $P$ 点的距离。基于上述几何关系,给出以下计算距离 $d$ 的公式(1)。在公式(1)中,$f$ 是相机的焦距; $\partial$ 是相机的倾斜角度;$h$ 是光学中心高度;($X_0$,$Y_0$)是指像平面和光轴的交点;
($X$,$Y$)是指在图像平面上的点P的投影。

假设$O_1$($U_0$,$V_0$)是光轴和图像平面的交点的照相机坐标,还假设对应于$x$轴和$y$轴在像平面上一个像素的物理尺寸是$dx$和$dy$。
然后:

$v$是在相机在$y$轴坐标,可以从目标检测过程中得到。所有其他的参数均是相机的内在参数,可以从相机矩阵检索。

另外OpenCV提供了\href{http://opencv-python-tutroals.readthedocs.io/en/latest/py_tutorials/py_calib3d/py_calibration/py_calibration.html}{相机标定}功能。校正后将会返回一个500万像素的树莓派相机矩阵。理想的情况是,$a\_x$ 和$a\_y$ 有相同的值。这两个值的方差将导致在图像中的非平方像素。

相机矩阵将返回像素值和$h$值。通过应用公式(3),将会得到实际的物理距离。

\subsubsection*{小车控制单元}
此项目中使用的小车有一个 开/关 类型控制器。当按下按钮时,相关芯片引脚和地之间的电阻为零。因此,使用 Arduino 电路板用来模拟按钮按下操作。选择四个 Arduino 针脚分别连接控制器四个芯片引脚上,分别对应于正向、 反向、 左和右的动作。一方面 Arduino引脚发送 LOW 信号显示控制器芯片的接地引脚;另一方面发送 HIGH 信号表明芯片引脚之间的电阻和地面保持不变。Arduino 通过 USB 连接到计算机。计算机使用串口向 Arduino 发送命令,然后 Arduino 读取命令,写出 LOW 或 HIGH 信号,模拟通过按下按钮来驾驶小车。




\end{document}